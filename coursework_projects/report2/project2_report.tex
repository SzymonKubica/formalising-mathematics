\documentclass[11pt]{article}
%% Package imports
\usepackage[newfloat]{minted}
\usepackage[utf8]{inputenc}
\usepackage{amsmath}
\usepackage{subcaption}
\usepackage{caption}
\usepackage{amsfonts}
\usepackage{amssymb}
\usepackage{physics}
\usepackage{graphicx}
\usepackage[left=2cm,right=2cm,top=2cm,bottom=2cm]{geometry}
\usepackage{multirow}
\usepackage{booktabs}
\usepackage{float}
\usepackage{verbatim}
\usepackage{amsthm}
\usepackage{fancyhdr}
\usepackage[dvipsnames]{xcolor}
\usepackage{parskip}
\usepackage{minted}
\usepackage{newunicodechar}
\usepackage{blindtext}
\usepackage{hyperref}
\hypersetup{
    colorlinks=true,
    linkcolor=blue,
    filecolor=magenta,
    anchorcolor=cyan,
    urlcolor=cyan,
    pdfpagemode=FullScreen,
    }
\renewcommand{\baselinestretch}{1.5}

\newenvironment{code}{\captionsetup{type=listing}}{}
\SetupFloatingEnvironment{listing}{name=Listing}

%% Commands for inserting big braces.
\newcommand\lb{\left\lbrace}
\newcommand\rb{\right\rbrace}

%% Commands for set such that notation
\newcommand\st{\text{ } | \text{ }}
\newcommand\sthat{\text{ such that }}

\definecolor{bg}{RGB}{240, 240, 240}

%% Math symbols
\newcommand\Q{\mathbb{Q}}
\newcommand\R{\mathbb{R}}
\newcommand\N{\mathbb{N}}
\newcommand\C{\mathbb{C}}
\newcommand\Lmu{\mathcal{L}^1(\mu)}
\newcommand\A{\mathcal{A}}

%% Shortcuts for inserting common words in the maths mode.
\newcommand\sucht{\text{ such that }}
\newcommand\weh{\text{ we have }}
\newcommand\as{\text{ as }}
\newcommand\sep{\text{ }}
\newcommand*\diff{\mathop{}\!\mathrm{d}}

\newunicodechar{ℝ}{\ensuremath{\mathbb{R}}}
\newunicodechar{ℕ}{\ensuremath{\N}}
\newunicodechar{ε}{\ensuremath{\varepsilon}}
\newunicodechar{μ}{\ensuremath{\mu}}
\newunicodechar{δ}{\ensuremath{\delta}}
\newunicodechar{∀}{\ensuremath{\forall}}
\newunicodechar{∃}{\ensuremath{\exists}}
\newunicodechar{∉}{\ensuremath{\not\in}}
\newunicodechar{𝓝}{\ensuremath{\cal{N}}}
\newunicodechar{∫}{\ensuremath{\int}}
\newunicodechar{λ}{\ensuremath{\lambda}}

\newunicodechar{₊}{\ensuremath{_+}}
\newunicodechar{↔}{\ensuremath{\iff}}
\newunicodechar{ₙ}{\ensuremath{_n}}
\newunicodechar{⁻}{\ensuremath{^-}}
\newunicodechar{⦃}{\ensuremath{\lbrace}}
\newunicodechar{⦄}{\ensuremath{\rbrace}}
\newunicodechar{≠}{\ensuremath{\ne}}
\newunicodechar{≥}{\ensuremath{\ge}}
\newunicodechar{∞}{\ensuremath{\infty}}
\newunicodechar{ᵐ}{\ensuremath{^m}}
\newunicodechar{∂}{\ensuremath{\partial}}
\newunicodechar{ℒ}{\ensuremath{\mathcal{L}}}


\newunicodechar{∈}{\ensuremath{\in}}
\newunicodechar{∧}{\ensuremath{\land}}
\newunicodechar{≤}{\ensuremath{\le}}
\newunicodechar{∨}{\ensuremath{\lor}}
\newunicodechar{⊆}{\ensuremath{\subset}}

\usepackage[shortlabels]{enumitem}

\newtheorem*{theorem}{Theorem}
\renewcommand\qedsymbol{$\blacksquare$}

%% Page style settings
\pagestyle{fancy}
\fancyfoot{}
\fancyhead[L]{\slshape{Formalising Mathematics - Coursework 2}}
\fancyhead[R]{\slshape{CID: 01871147}}
\fancyfoot[C]{\thepage}
\begin{document}

\title{Formalising Mathematics - Coursework 2 \\ Vitali's Theorem}
\date{\today}
\author{CID: 01871147}
\maketitle

\section*{Introduction}

As the second part of my coursework for the module I've decided to formalise the Vitali's theorem
which is taught in the second year measure theory course: MATH50006:
Lebesgue Measure and Integration. I made this decision because I thoroughly enjoyed
learning about measure theory in my second year. On top of that, formalising measure
theoretic concepts was somewhat similar to the process of formalising theorems
from Analysis I covered in the first part of my coursework.
The similarity was in the structure of the proof; Vitali's theorem requires using
$\varepsilon-\delta$ reasoning which I was familiar with after completing my first
coursework. However, measure theory is also sufficiently different from analysis
to make it very interesting to work with, provide an optimal level of challenge
and expand my knowledge of the Lean programming language.

This report documents the process of formalising Vitali's theorem using the Lean
programming language. It is a functional language which can also be used as an interactive
theorem prover. I used Lean together with the mathlib library which contains many
fundamental theorems and identities that are useful when building more complex proofs.

After formalising two large proofs using this toolkit, I learned to appreciate that
the mathlib library shouldn't be thought of as merely an extension of the basic
capabilities of the language. Instead, mathlib provides all of the API's with which
are needed when proving mathematical theorems. Without this library,
proving complicated statements such as the Vitali's theorem would require us to
define all of the measure theory leading up to the statement of the theorem
which would be quite a lentghy process.

\section*{Proof of the Vitali's Theorem}
Before I document the process of formalising, let us first state the
theorem and observe the proof which I followed when translating the theorem
into Lean.

\begin{theorem}[Vitali's Theorem]
  Let $X$ be a measurable space and $\mu$ be a measure on it. Assume $\mu(X)<\infty$.
  Let $f, f_n : X \to \R, n \in \N$ be functions all in $\Lmu$. Then
  the following are equivalent:
\end{theorem}
    \hspace{1cm} (T1) $f_n \xrightarrow{\mu} f $ and $ \mathcal{F} = \lbrace f_n : n \in \N\rbrace$ has uniformly absolutely continuous
     integrals,

    \hspace{1cm} (T2) $\int |f_n - f| \diff\mu \to 0$, as $n \to \infty$.

Note that a family $\mathcal{F} \subset \Lmu$ of functions
has uniformly absolutely continuous integrals if
\begin{equation}
  \forall \varepsilon > 0, \sep \exists \delta > 0, \sthat  \forall f \in \mathcal{F},
  \sep \forall A \in \mathcal{A}, \weh \mu(A) < \delta \implies \int_A |f|\diff \mu < \varepsilon.
\end{equation}

Where $\A$ is the sigma algebra on $X$ corresponding to $\mu$.

Let us start by showing he backwards direction of the theorem i.e. that
$(\text{T}2) \implies (\text{T}1)$. Let us assume that  $f_n$ converges to  $f$ in  $\Lmu$.
We'll first focus on showing that  $f_n \xrightarrow{\mu} f$.

By definition, $f_n$ converges to  $f$ in measure if:
  \[
    \forall \varepsilon > 0,\text{ }  \mu \left(\lbrace x \in X \st
    \varepsilon \le |f_n(x) - f(x)|\rbrace\right) \to 0 \text{ as } n \to \infty
 .\]
Let $\varepsilon > 0$ be arbitrary, now by applying Markov's inequality we have:
\[
 \forall n \in \N, \text{ } 0 \le \mu \left(\lbrace x \in X \st
\varepsilon \le |f_n(x) - f(x)|\rbrace\right) \le \frac{1}{\varepsilon} \int |f_n - f|
\diff\mu
.\]
Now by our assumption of convergence in $\Lmu$ we deduce that
\[
\frac{1}{\varepsilon}\int |f_n - f| \diff\mu \to 0 \as n \to \infty
.\]

Hence, by applying the squeeze theorem we conclude that $f_n \xrightarrow{\mu} f$.

Now let us show that the family $\mathcal{F} = \lbrace f_n \st n \in \N\rbrace $ has uniformly
absolutely continuous integrals, i.e. that (1) holds for $\mathcal{F}$.
Let $\varepsilon > 0$ be arbitrary, by convergence in $\Lmu$ we can find  $n_0 = n_0(\varepsilon)$
satisfying :
\[
\forall n \ge n_0, \weh \int |f_n - f| \diff\mu < \frac{\varepsilon}{2}
.\]
Let us consider the finite family $\mathcal{F}' = \lbrace f, f_1, f_2, \ldots, f_{n_0}\rbrace $,
using the fact that any finite family has uniformly absolutely continous integrals,
we can find a $\delta > 0$, such that:
\begin{equation}
  \forall \text{ } A \in \A \text{ with } \mu(A) < \delta, \weh \int_A |f| \diff\mu < \frac{\varepsilon}{2}
  \text{ and } \max_{n \le n_0}\int_A |f_n| \diff\mu < \frac{\varepsilon}{2}.
\end{equation}

Now consider the following chain of inequalities for $n > n_0,\sep A \in \A,\sep \mu(A) < \delta$:
\begin{equation}
\int_A |f_n| \diff\mu \le \int_A |f| \diff\mu + \int_A |f_n - f| \diff\mu \le \int_A |f| \diff\mu + \int |f_n - f| \diff\mu. < \frac{\varepsilon}{2} + \frac{\varepsilon}{2} = \varepsilon.
\end{equation}
From the second part of (2) we can deduce that:
\begin{equation}
\max_{n \le n_0}\int_A |f_n| \diff\mu < \frac{\varepsilon}{2} < \varepsilon.
\end{equation}
Therefore, by combining the results (3) and (4) above we cover all $n \in \N$, hence
we conclude that:
\[
\forall n \in \N, \int_A |f_n| \diff\mu < \varepsilon
.\]

Therefore we have shown that $\mathcal{F}$ has uniformly absolutely continuous
integrals and consequently the backwards direction of the theorem holds.

Now we'll show that $(\text{T}1) \implies (\text{T}2)$.
Let us assume that $f_n$ converges to $f$ in measure and that the family $\mathcal{F}$
has uniformly absolutely continuous integrals. We'll prove that
$\int |f_n - f| \diff\mu \to 0$, as $n \to \infty$ using the fact that a sequence
converges 0 if every subsequence has a subsequence which converges to 0.
Let $(f_{n_k})_{k\in\N}$ be an arbitrary subsequence of  $(f_{n})_{n\in\N}$.
Consequently, $(\int |f_{n_k} - f| \diff\mu)_{k\in\N}$ is a subsequence of
$(\int |f_{n} - f| \diff\mu)_{n\in\N}$. We'll show that it has a subsequence such that:
\[
  \int |f_{i} - f| \diff\mu \to 0 \as i \to \infty \text{ along } \Lambda
.\]
Which will in turn imply that $\int |f_{n} - f| \diff\mu \to 0 \as n \to \infty $
because $n_k$ was an arbitrary subsequence.

Let us observe that, since $f_n \xrightarrow{\mu} f$ and $(f_{n_k})_{k\in\N}$
is a subsequence of  $(f_{n})_{n\in\N}$, it implies that we also have:
\[
f_{n_k} \xrightarrow{\mu} f
.\]
Also given the fact that we have $\mu(X) < \infty$, we can apply the proposition
(2.45) from the lecture notes for  MATH50006: Lebesgue Measure and Integration, which
tells us that there exists a subsequence $\Lambda \subset n_k$ such that:
\begin{equation}
  f_i \to f \text{ }\mu\text{-a.e.} \as n \in \Lambda \to \infty
\end{equation}

Furthermore, given that $\mathcal{F}$ has uniformly absolutely continuous integrals,
we can pass to a subsequence and deduce that  $\mathcal{F}'' = \lbrace f_i \st i \in \Lambda \rbrace$
also has uniformly absolutely continuous integrals. That follows from the definition
(1), if for and $\varepsilon$ we get a  $\delta$ such that:
\[
\forall f \in \mathcal{F},
  \forall A \in \mathcal{A}, \weh \mu(A) < \delta \implies \int_A |f|\diff \mu < \varepsilon.
.\]

Then also, since $\mathcal{F}'' \subset \mathcal{F}$, we get that
\begin{equation}
\forall f \in \mathcal{F''},
  \forall A \in \mathcal{A}, \weh \mu(A) < \delta \implies \int_A |f|\diff \mu < \varepsilon
\end{equation}
Now we use (3), (4) and the proposition that if a sequence of functions converges $\mu$-a.e.
and has uniformly absolutely continuous integrals, then it converges in  $\Lmu$.
Therefore we have shown:
\[
  \int |f_{i} - f| \diff\mu \to 0 \as i \to \infty \text{ along } \Lambda
.\]
Which concludes the proof as we have shown that an arbitrary subsequence of
$(\int |f_{n} - f| \diff\mu)_{n\in\N} $ has a subsequence convergent to 0. \hfill \qedsymbol

\section*{The Process of Formalising}
\subsection*{Methodology}

In order to formalise Vitali's theorem, I followed a similar approach to what
I did in my first project. This time however, I was able to avoid some of the mistakes that
I made in my first project. The first modification was that I started proving the
theorem by first breaking it down into separate lemmas which encapsulated the
three 'branches' that need to be shown in order to prove Vitali's theorem. In the proof
presented above, we first needed to show that
convergence in $\Lmu$ implies convergence in measure (first 'branch'), then we have shown that
it implies that  $\mathcal{F}$ has uniformly absolutely continuous integrals. Finally,
we have shown the forward direction of the theorem which concluded the proof.
The corresponding logical structure of the proof can be seen in the listing below.

\begin{code}
\begin{minted}[numbersep=5pt, fontsize=\small, linenos=false, bgcolor=bg]{lean}
theorem vitali_theorem {m : measurable_space X} {μ : measure X} [is_finite_measure μ]
(f : ℕ → X → ℝ) (g : X → ℝ) (hf : ∀ (n : ℕ), (f n) ∈_L1{μ}) (hg : g ∈_L1{μ}) :
f -→ g in_measure{μ} ∧ unif_integrable f 1 μ ↔ f -→ g in_L1{μ} :=
begin
  split,
  { rintro ⟨h_tendsto_μ , h_unif_int⟩,
    exact tendsto_L1_of_unif_integr_of_tendsto_in_μ hf hg h_tendsto_μ  h_unif_int, },
  { intro h_tendsto_L1,
    split,
    { exact tendsto_in_measure_of_tendsto_L1 hf hg h_tendsto_L1 },
    { exact unif_integrable_of_tendsto_L1 hf hg h_tendsto_L1, }, },
end
\end{minted}
\captionof{listing}{Main statement of Vitali's theorem.}
\end{code}


In my implementation, instead of starting to prove the theorem
within one monolithic proof (which admittedly was my biggest mistake in my first project),
I started by defining sub-lemmas for each of those three branches and working out
the precise sets of hypotheses that they required to be shown.

Modularising the argument from the very beginning contributed to an improved experience
of using the language. Because my proofs were shorter, they were compiled much
faster which allowed me to get interactive feedback on the state of my proof a lot
quicker and perform faster iterations if something wasn't quite working.
As I prioritised introducing lemmas, I
was trying to generalise some of the intermediate propositions that I was proving,
which often allowed me to reuse one lemma multiple times. That modular approach
also helped me make progress when I got stuck on something which I couldn't prove,
in such case I would almost always introduce a lemma containing a generalised version
of the problematic proposition, put a \texttt{sorry} tactic inside of it to make it
compile, and then continue with the rest of the proof.

The second modification to my methodology was that I needed to start
interacting with the API provided by the mathlib library more than I used to in
my first project. This change was because in this project my goal was to prove
a statement which involved advanced measure theory. I could no longer simply define
my own definition of continuity as I did in my first project, this
time it wasn't feasible to define my own notions of a measure, measurable
functions or the three notions of convergence that I operated on.

A good example of that approach can be seen below, where in the process of proving
that convergence in $\Lmu$ implies convergence in measure
I needed to use Markov's inequality and relied on the implementation provided by the library.

\begin{code}
\begin{minted}[numbersep=5pt, fontsize=\small, linenos=false, bgcolor=bg]{lean}
theorem tendsto_in_measure_of_tendsto_L1 {m : measurable_space X} {μ : measure X}
{f : ℕ → X → ℝ} {g : X → ℝ} (hf : ∀ (n : ℕ), f n ∈_L1{μ}) (hg : g ∈_L1{μ})
(h : f -→ g in_L1{μ}) : f -→ g in_measure{μ} :=
begin
  -- Here we aim to apply the Markov's inequality to fₙ - g for all n ∈ ℕ.
  have h2 : ∀ (ε : ℝ≥0∞), ε ≠ 0 →  ∀ (n : ℕ), μ { x | ε ≤ ‖f n x - g x‖₊ } ≤
    ε⁻¹  * ∫|(f n) - g|dμ,
  { intros ε hε n,
    -- We use sub_strongly_measruable to show that fₙ - g is ae_strongly_measurable.
    exact markov_ineq_L1 μ (sub_strongly_measurable (hf n) hg) ε hε, },
  simp at h2,
\end{minted}
\captionof{listing}{Using mathlib's Markov inequality.}
\end{code}

\subsection*{Challenges}

As the scope of this project included interacting with more advanced theory
from second year university mathematics, formalising it has proven more difficult
than the first year analysis that I was working on in my first project. The two
interesting challenges that I faced were type coercions and proving
statements involving finite objects.

The first one of these came up because I
was dealing with measure theory, and since we want a measure to be a function which
can return infinity, we need to work with extended non-negative real numbers.
In usual mathematical papers it is not that difficult to handle it, we just declare
those numbers to be defined as $\R^+ \cup \lbrace+\infty\rbrace$ and require a
certain degree of caution when dealing with the 'infinity' element. However, in
Lean the extended non-negative real numbers (\texttt{ennreal}) are a completely
different type from the usual real numbers. Therefore, if we for instance require
that $\mu (A) < \varepsilon $ for some  $\varepsilon \in \R $ and  $A \in \A$, that
proposition doesn't even make sense in the world of Lean. That is because the return
type of $\mu$ is \texttt{ennreal} whereas  $\varepsilon$ is an ordinary real number. Because
of this, in the code you need to convert types of variables using special functions
which are called coercions. The difficult thing about this is that not always is
it possible to coerce between types. For instance it doesn't make sense to convert
a negative number into an  \texttt{ennreal}. Because of this, often when converting
between these types one has to supply additional hypotheses such as that an
\texttt{ennreal} number is not equal to infinity if we want to convert it into
a real number. An example of the usage of coercions can be seen in the listing below.
\begin{code}
\begin{minted}[numbersep=5pt, fontsize=\small, linenos=false, bgcolor=bg]{lean}
theorem ennreal_squeeze_zero {a b : ℕ → ℝ≥0∞} (ha : ∀ (n : ℕ), a n < ∞)
(hb : ∀ (n : ℕ), b n < ∞) (h0b : ∀ n, 0 ≤ b n) (hba : ∀ n, b n ≤ a n)
(ha_to_0 : tendsto a at_top (𝓝0)) : tendsto b at_top (𝓝0) :=
begin -- First we need to coerce the sequences to apply nnreal_squeeze_zero on them.
  lift a to ℕ → ℝ≥0, {exact lift_ennreal_to_nnreal ha },
  lift b to ℕ → ℝ≥0, {exact lift_ennreal_to_nnreal hb },
  rw [← ennreal.coe_zero, ennreal.tendsto_coe] at *, -- removes coercions from ℝ≥0 to ℝ≥0∞
  simp at hba,
  dsimp at h0b,
  have h0b : ∀ (n : ℕ), 0 ≤ b n,
  { intro n,-- Here we need to rewrite so that h0b matches hypothesis of nnreal_squeeze_zero
    exact ennreal.coe_le_coe.mp (h0b n), },
  exact nnreal_squeeze_zero h0b hba ha_to_0,
end
\end{minted}
\captionof{listing}{Coercions required to apply \texttt{squeeze\_zero}}
\end{code}
It is a proof of the squeeze theorem for \texttt{ennreal} numbers, it was required
by the main proof to apply squeeze theorem to two sequences of ennreal numbers. In order
to do this, I needed to cast those sequences into non-negative real numbers,
then perform coercions on the limits to make the types match up and finally
I was able to apply the library version of the squeeze theorem for non-negative
real numbers. It illustrates how coercions make the proofs much more complicated
as the programmer needs to ensure that each variable if of precisely the right
type when it is in an expression involving terms of the specific type.

The second aspect of the proof that I found more difficult than expected was
proving propositions for finite objects such as lists and sets. For instance,
when proving that a singleton set $\lbrace g\rbrace$ containing a single measurable
function  $g \in \Lmu $ has uniformly absolutely continuous integrals was more
involved than I anticipated. The proof of that claim can be seen in Listing 4 below.
\begin{code}
\begin{minted}[numbersep=5pt, fontsize=\small, linenos=false, bgcolor=bg]{lean}
/-- This lemma proves that if we take a single function in L1(μ) then this set
    has uniformly absolutely continuous integrals. -/
theorem unif_integrable_singleton {m : measurable_space X} {μ : measure X}
{g : X → ℝ} (hg :g ∈_L1{μ}) :
∀ ⦃ε : ℝ⦄ (hε : 0 < ε), ∃ (δ : ℝ) (hδ : 0 < δ), ∀ s, measurable_set s
  → μ s ≤ ennreal{δ} → (∫_{s}_|g|dμ) ≤ ennreal{ε} :=
begin
  -- We define a finite set {g}.
  let G : fin 1 → X → ℝ := λ i, g,
  have hG : ∀ (i : fin 1), (G i) ∈_L1{μ}, { intro i, exact hg, },
  have hG_unif_integrable : unif_integrable G 1 μ,
  -- Here we show that it is has uniformly abs. cont. integrals.
  { exact unif_integrable_fin μ (le_refl 1) ennreal.one_ne_top hG, },
  intros ε hε,
  specialize hG_unif_integrable hε,
  rcases hG_unif_integrable with ⟨δ, hδ, hG⟩,
  use δ,
  split,
  { exact hδ },
  { intros s hs,
    specialize hG 1 s hs,
    exact hG, },
end
\end{minted}
\captionof{listing}{Proof of a proposition for a singleton set.}
\end{code}

As you can see above, in order to show that proposition I needed to define a new
'dummy' function \texttt{G} which maps from \texttt{fin 1} which is a finite sequence
of natural numbers up to and not including 1 (i.e. it is the set $\lbrace 0 \rbrace$.)
Then I used the \texttt{unif\_integrable\_fin} theorem from the library which shows that
any finite family of functions has uniformly absolutely continuous integrals.

Another interesting lemma which I have introduced aimed to show that if we have
a list of two strongly measurable functions $f$ and  $g$ then if we define a list
$[f, g]$ then all items in that list are strongly measurable. When formulated using
plain English, this propositions sound trivial, however in lean it required some
careful case analysis based on whether the item is equal to the first or second
item in the list. The reason why I introduced this lemma was that I needed to show
that if $f - g$ is strongly
measurable. Hence I needed to form a finite list $[f, -g]$ and then apply the
proposition \texttt{ae\_strongly\_measurable\_sum}  which shows that a sum of a finite
list of strongly measurable functions is strongly measurable. However,
after proving it I realised that it is actually a known fact in the
library (called \texttt{ae\_strongly\_measurable.sub}) and I didn't need to do
all of the list manipulations. In the end, I decided to leave that proof in my implementation
because I think it is an interesting exercise illustrating proofs with finite sets.
\begin{code}
\begin{minted}[numbersep=5pt, fontsize=\small, linenos=false, bgcolor=bg]{lean}
lemma list_elem_ae_strongly_measurable {m : measurable_space X} {μ : measure X}
{f g : X → ℝ} (hf : ae_strongly_measurable f μ) (hg : ae_strongly_measurable g μ)
{l : list (X → ℝ)} (hl : l = [f, g]) :
∀ (k : X → ℝ), k ∈ l → ae_strongly_measurable k μ :=
begin
  intros k k_in_l,
  by_cases k = f,
  { rw h, exact hf},
  { -- Here we show that if it is not equal to f then it must be equal to g
    -- as l has only two elements. As stated above it wasn't trivial.
    have hkg : k = g,
    { have h_cases : k = f ∨ k ≠ f ∧ k ∈ [g],
      { rw hl at k_in_l, exact list.eq_or_ne_mem_of_mem k_in_l, },
      cases h_cases,
      { exfalso, exact h h_cases, },
      { exact list.mem_singleton.mp h_cases.right }, },
    rw hkg,
    exact hg, },
end
\end{minted}
\captionof{listing}{Proof of a proposition for a finite list.}
\end{code}

\subsection*{Language Extensions}

One thing that I noticed when iterating with the API provided by mathlib is that
even though it is very exhaustive and almost any imaginable theorem from measure
theory is already available there, it sometimes isn't very easy to read
the code because of the verbosity of the API that the library provides.

Consider the following example: let $\varepsilon \in \R, \sep s \in \A$ and $f$ be
any integrable function, suppose I wanted to express the following inequality
using mathlib:
\[
\int_s |f|\diff\mu \le \varepsilon
.\]

In my proofs I would have to write this:
\begin{code}
\begin{minted}[numbersep=5pt, fontsize=\small, linenos=false, bgcolor=bg]{lean}
snorm (s.indicator (f)) 1 μ ≤ ennreal.of_real (ε)
\end{minted}
\captionof{listing}{Limited expressiveness of the API.}
\end{code}

Although it might not look too overwhelming, if there are multiple integrals
involved, things can get out of control quite quickly . Another problem
that I identified is that in the signature of the \texttt{snorm} function
above, one has to write $1 \sep \mu $ at the end which specifies that we want
to use the $\Lmu $ norm. It might not be a very big issue, however I kept
forgetting the order in which those two need to be specified. Moreover, the
fact that these trailing arguments are a common occurrence among many functions
in the library didn't make it easier. One has to remember all of the function
signatures or look them up constantly which slows down the progress. The
listing below illustrates how the verbose syntax impact readability.
\begin{code}
\begin{minted}[numbersep=5pt, fontsize=\small, linenos=false, bgcolor=bg]{lean}
lemma extract_δ_uaci {m : measurable_space X} {μ : measure X} [is_finite_measure μ]
{ f : ℕ → X → ℝ } { g : X → ℝ }
(h_f_n_uaci : ∀ (ε : ℝ), ε > 0 →  ∃ (δ : ℝ) (hδ : 0 < δ), ∀ (n : ℕ) s, measurable_set s →
  μ s ≤ ennreal.of_real δ → snorm (s.indicator (f n)) 1 μ ≤ ennreal.of_real (ε / 3))
(h_g_uaci : ∀ (ε : ℝ), ε > 0 →  ∃ (δ : ℝ) (hδ : 0 < δ), ∀ s, measurable_set s →
  μ s ≤ ennreal.of_real δ → snorm (s.indicator (g)) 1 μ ≤ ennreal.of_real (ε / 3))
: ∀ (ε : ℝ), ε > 0 →  ∃ (δ : ℝ) (hδ : 0 < δ),
  ∀ s, measurable_set s → μ s ≤ ennreal.of_real δ →
  ∀ (n : ℕ), snorm (s.indicator (f n)) 1 μ ≤ ennreal.of_real (ε / 3) ∧
  snorm (s.indicator (g)) 1 μ ≤ ennreal.of_real (ε / 3) :=
begin ...
\end{minted}
\captionof{listing}{Verbosity of the mathlib functions.}
\end{code}

In order to solve the problem of readability, I decided to add new notation which
was specific to the file that I was working in. It is important to note that
overriding notation with new definitions should be done with caution and in a
consistent manner. That's why the professionals maintaining the mathlib library probably
wouldn't opt for my custom definitions which are far from being perfect. I introduced
the new notation only because it allowed me to explore some of the more advanced
features of the language and make the code development process slightly more
convenient. I wanted to indicate that I don't claim that the API in the mathlib
library is flawed, my definitions were merely an exploration and shouldn't be
considered as a claim that the library is not good enough.

I began by introducing alias definitions and new notation for the basic measure theoretic concepts such
as the notion of being in $\Lmu$ and the three types of convergence:
\begin{code}
\begin{minted}[numbersep=5pt, fontsize=\small, linenos=false, bgcolor=bg]{lean}
def tendsto_in_L1 {m : measurable_space X}
(μ : measure X) (f : ℕ → X → ℝ) (g : X → ℝ) : Prop :=
filter.tendsto (λ (n : ℕ), snorm (f n - g) 1 μ) filter.at_top (𝓝0)

def tendsto_μ_ae {m : measurable_space X} (μ : measure X)
(f : ℕ → X → ℝ) (g : X → ℝ) : Prop :=
∀ᵐ (x : X) ∂μ, filter.tendsto (λ (n : ℕ), f n x) filter.at_top (𝓝(g x))
-- Now three notions of convergence can be expressed using a consistent notation.
notation f `-→` g `in_L1{` μ `}` := tendsto_in_L1 μ f g
notation f `-→` g `in_measure{` μ `}` := tendsto_in_measure μ f at_top g
notation f `-→` g `{` μ `}_a-e` := tendsto_μ_ae μ f g
-- Also the notion of being a member of L1 is somewhat hard to decipher.
notation f `∈_L1{` μ `}` :=  mem_ℒp f 1 μ
-- Notation for snorms
notation `∫|` f `|d` μ :=  (snorm (f) 1 μ)
notation `∫_{` s `}_|` f `|d` μ :=  snorm (s.indicator f) 1 μ
-- Notation for converting between types
notation `ennreal{` a `}` := ennreal.of_real(a)
notation `real{` a `}` := ennreal.to_real(a)
\end{minted}
\captionof{listing}{New notation definitions.}
\end{code}

After that I replaced all occurrences of the previous notation with the custom
definitions. The result of that refactoring, which can be seen in the listing below,
can be contrasted with Listing 7. As we can see the expressions have become much
shorter and arguably more readable assuming that one knows that there are some
notations overrides in place.
\begin{code}
\begin{minted}[numbersep=5pt, fontsize=\small, linenos=false, bgcolor=bg]{lean}
lemma extract_δ_uaci {m : measurable_space X} {μ : measure X} [is_finite_measure μ]
{ f : ℕ → X → ℝ } { g : X → ℝ }
(h_f_n_uaci : ∀ (ε : ℝ), ε > 0 →  ∃ (δ : ℝ) (hδ : 0 < δ), ∀ (n : ℕ) s, measurable_set s →
              μ s ≤ ennreal{δ} → ( ∫_{s}_|f n|dμ ) ≤ ennreal{ε / 3})
(h_g_uaci : ∀ (ε : ℝ), ε > 0 →  ∃ (δ : ℝ) (hδ : 0 < δ), ∀ s, measurable_set s →
            μ s ≤ ennreal{δ} → ( ∫_{s}_|g|dμ ) ≤ ennreal{ε / 3})
: ∀ (ε : ℝ), ε > 0 →  ∃ (δ : ℝ) (hδ : 0 < δ), ∀ s, measurable_set s → μ s ≤ ennreal{δ} →
  ∀ (n : ℕ), ( ∫_{s}_|f n|dμ ) ≤ ennreal{ε / 3} ∧ ( ∫_{s}_|g|dμ ) ≤ ennreal{ε / 3} :=
begin ...

theorem tendsto_in_measure_of_tendsto_L1 {m : measurable_space X} {μ : measure X}
{f : ℕ → X → ℝ} {g : X → ℝ} (hf : ∀ (n : ℕ), f n ∈_L1{μ}) (hg : g ∈_L1{μ})
(h : f -→ g in_L1{μ}) : f -→ g in_measure{μ} := -- <- New notation for convergence.
begin ...
\end{minted}
\captionof{listing}{Code after introducing new notation.}
\end{code}

The impact of the applied refactoring might not be apparent at the first glance,
especially given how complex the definition above still is. However, I find it
more readable and a bit faster to type out. The new definitions that I
found particularly convenient were the ones for convergence $\mu$-a.e, in measure
and in  $\Lmu$. It can be seen in the second part of the Listing 9 that now the
API for convergence is consistent in that you need to specify that $f \to g$ and
then state in what way.


\section*{Conclusions}

To conclude, the second part of my coursework was a great opportunity to learn
more about proving theorems in Lean and also revise some concepts from my
favourite second year module.

Having proved Vitali's theorem I have learned
how to use the toolkit provided by mathlib in the context of measure theory. I also
gained appreciation for the usefulness of the library and how indispensable it
is when proving more advanced propositions. I realised that it is important to be
able to search through the library efficiently and quickly identify if a given
theorem in the library can be applied in the context of the proof that one is
currently working on.

The project also allowed me to explore some of the more advanced features of the
Lean programming language. Introducing the new notation was an interesting way to
learn how prefix and infix operators are handled in lean and how overriding operators
can be used to augment the syntax of the language.

\begin{comment}

-- Good example of notation refactoring

\end{comment}





\end{document}
