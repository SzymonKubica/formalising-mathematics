\documentclass[11pt]{article}
%% Package imports
\usepackage[newfloat]{minted}
\usepackage[utf8]{inputenc}
\usepackage{amsmath}
\usepackage{subcaption}
\usepackage{caption}
\usepackage{amsfonts}
\usepackage{amssymb}
\usepackage{physics}
\usepackage{graphicx}
\usepackage[left=2cm,right=2cm,top=2cm,bottom=2cm]{geometry}
\usepackage{multirow}
\usepackage{booktabs}
\usepackage{float}
\usepackage{verbatim}
\usepackage{amsthm}
\usepackage{fancyhdr}
\usepackage[dvipsnames]{xcolor}
\usepackage{parskip}
\usepackage{minted}
\usepackage{newunicodechar}
\usepackage{blindtext}
\usepackage{hyperref}
\hypersetup{
    colorlinks=true,
    linkcolor=blue,
    filecolor=magenta,
    anchorcolor=cyan,
    urlcolor=cyan,
    pdfpagemode=FullScreen,
    }
\renewcommand{\baselinestretch}{1.5}

\newenvironment{code}{\captionsetup{type=listing}}{}
\SetupFloatingEnvironment{listing}{name=Listing}

%% Commands for inserting big braces.
\newcommand\lb{\left\lbrace}
\newcommand\rb{\right\rbrace}

%% Commands for set such that notation
\newcommand\st{\text{ } | \text{ }}
\newcommand\sthat{\text{ such that }}

\definecolor{bg}{RGB}{240, 240, 240}

%% Math symbols
\newcommand\Q{\mathbb{Q}}
\newcommand\R{\mathbb{R}}
\newcommand\N{\mathbb{N}}
\newcommand\C{\mathbb{C}}
\newcommand\Lmu{\mathcal{L}^1(\mu)}
\newcommand\A{\mathcal{A}}

%% Shortcuts for inserting common words in the maths mode.
\newcommand\sucht{\sep \mathrm{such} \sep \mathrm{that} \sep}
\newcommand\weh{\text{ we have }}
\newcommand\as{\text{ as }}
\newcommand\sep{\text{ }}
\newcommand*\diff{\mathop{}\!\mathrm{d}}

\newunicodechar{ℝ}{\ensuremath{\mathbb{R}}}
\newunicodechar{ℕ}{\ensuremath{\N}}
\newunicodechar{ε}{\ensuremath{\varepsilon}}
\newunicodechar{μ}{\ensuremath{\mu}}
\newunicodechar{δ}{\ensuremath{\delta}}
\newunicodechar{∀}{\ensuremath{\forall}}
\newunicodechar{∃}{\ensuremath{\exists}}
\newunicodechar{∉}{\ensuremath{\not\in}}
\newunicodechar{𝓝}{\ensuremath{\cal{N}}}
\newunicodechar{∫}{\ensuremath{\int}}
\newunicodechar{λ}{\ensuremath{\lambda}}

\newunicodechar{₊}{\ensuremath{_+}}
\newunicodechar{↔}{\ensuremath{\iff}}
\newunicodechar{ₙ}{\ensuremath{_n}}
\newunicodechar{⁻}{\ensuremath{^-}}
\newunicodechar{⦃}{\ensuremath{\lbrace}}
\newunicodechar{⦄}{\ensuremath{\rbrace}}
\newunicodechar{≠}{\ensuremath{\ne}}
\newunicodechar{≥}{\ensuremath{\ge}}
\newunicodechar{∞}{\ensuremath{\infty}}
\newunicodechar{ᵐ}{\ensuremath{^m}}
\newunicodechar{∂}{\ensuremath{\partial}}
\newunicodechar{ℒ}{\ensuremath{\mathcal{L}}}


\newunicodechar{∈}{\ensuremath{\in}}
\newunicodechar{∧}{\ensuremath{\land}}
\newunicodechar{≤}{\ensuremath{\le}}
\newunicodechar{∨}{\ensuremath{\lor}}
\newunicodechar{⊆}{\ensuremath{\subset}}

\usepackage[shortlabels]{enumitem}

\newtheorem*{theorem}{Theorem}
\newtheorem*{lemma}{Lemma}
\renewcommand\qedsymbol{$\blacksquare$}

%% Page style settings
\pagestyle{fancy}
\fancyfoot{}
\fancyhead[L]{\slshape{Formalising Mathematics - Coursework 3}}
\fancyhead[R]{\slshape{CID: 01871147}}
\fancyfoot[C]{\thepage}
\begin{document}

\title{Formalising Mathematics - Coursework 3 \\ Banach-Steinhaus Theorem and Extensions}
\date{\today}
\author{CID: 01871147}
\maketitle

\section*{Introduction}

As the final part of my coursework I decided to formalise the Banach-Steinhaus theorem
which I've learned about this year when taking the functional analysis course:
MATH60030: Functional Analysis. Given that the final part of the coursework needed
to cover a theorem from the 3rd year of undergraduate mathematics, I had to pick
a theorem from Functional Analysis as it is the only other 3rd year mathematics
module that I do apart from Formalising Mathematics. However that constraint wasn't
a big issue for me as I thoroughly enjoyed taking the module during the Autumn Term
2022/2023.

What is more, I deliberately chose the past two projects so that the experience
that I gained by working on those was very helpful when working on this project.
My first project covered the intermediate value theorem and was a great way to
gain experience proving theorems in mathematical analysis which mainly focus on
manipulating inequalities and being able to estimate and bound expressions. The
second project - Vitali's theorem allowed me to learn how to properly use the mathlib
library and further developed my toolkit for proving claims in analysis.

This report documents the process of formalising Banach-Steinhaus theorem using the Lean programming language.
It is a functional language which can also be used as an interactive theorem prover. I used Lean together
with the mathlib library which contains many fundamental theorems and identities that are useful when
building more complex proofs.

In what follows we'll first state and prove the theorem. Then I will explain the
methodology that I followed in order to formalise the theorem. I will also describe
the interesting challenges that needed to be overcome when formalising the proof.
The last section contains conclusions which I made after working on the three sizeable
projects in Lean and describe the future things that I will be working on.

\section*{Proof of the Banach-Steinhaus Theorem}

Before I document the process of formalising, let us first state the
theorem and observe the proof which I followed when translating the theorem
into Lean.
\begin{theorem}[Banach-Steinhaus Theorem]
  Let $X, Y$ be normed vector spaces over $\R$, assume that $X$ is complete. Consider
  the following family of continuous linear operators:
  $\left( A_\lambda \right)_{\lambda \in \Lambda} \subset \mathcal{L}\left( X, Y \right) $
  Assume that this family is bounded pointwise i.e. (with each upper bound being dependent on $x$)
  \[
    \forall x \in X \sep \sup_{\lambda \in \Lambda} \|A_\lambda x\|_Y < \infty
  .\]
  Then $\left( A_\lambda \right)_{\lambda \in \Lambda}$ is bounded uniformly i.e.
  \[
    \sup_{\lambda \in \Lambda} \|A_\lambda \|_{\mathcal{L}\left( X, Y \right) } < \infty
  .\]
\end{theorem}
Before we prove the theorem let us first reformulate it in terms of upper
bounds instead of requiring that suprema are finite. The reasoning behind doing
this is that the statement of the theorem below is equivalent to the one above,
however it is much more convenient to work with in Lean. That is because
bounding a specific term is easier than working with the finiteness suprema. In
fact the two alternative ways of formulating the theorem are present in the
mathlib library, however the original statement that can be seen above is
considered to be an alternative definition.

\begin{theorem}[Banach-Steinhaus Theorem (Alternative formulation)]
  Let $X, Y$ be normed vector spaces over $\R$, assume that $X$ is complete. Consider
  the following family of linear operators:
  $\left( A_\lambda \right)_{\lambda \in \Lambda} \subset \mathcal{L}\left( X, Y \right) $
  Assume that this family is bounded pointwise i.e.
  \[
    \forall x \in X, \sep \exists \sep K_x \in \R \sucht \forall \lambda \in \Lambda \sep \|A_\lambda x\|_Y \le K_x
  .\]
  Then $\left( A_\lambda \right)_{\lambda \in \Lambda}$ is bounded uniformly i.e.
  \[
    \exists K' \in \R \sucht \forall \lambda \in \Lambda \sep \|A_\lambda \|_{\mathcal{L}\left( X, Y \right) } \le K'  .\]
\end{theorem}

later we'll see how in Lean
we can use the proof of the alternative version to easily deduce the theorem in it's
initial form. Before we do that however, let us consider the following lemma which
is a consequence of the Baire Category Theorem and allows us to deduce topological
properties of a family of continuous functions which are bounded pointwise.

\begin{lemma}
  Let $\left( X, d \right) $ be a complete metric space. Let
  $\left( f_\lambda \right)_{\lambda \in \Lambda} $ be a family of continuous
  functions $$f_\lambda : X \to \R $$ (note that they don't need to be linear).
  If the family above is bounded pointwise i.e. :
  \[
    \forall x \in X, \sep \exists \sep K_x \in \R \sucht \forall \lambda \in \Lambda \sep |f_\lambda(x)| \le K_x
  .\]
  Then there exists an open ball $B_r(x_0)$ with  $x_0 \in X$ and  $0 < r \in \R$.
  Such that $\left( f_\lambda \right)_{\lambda \in \Lambda} $ is uniformly bounded
  on it, that is:
  \[
    \exists K' \in \R \sucht \forall x \in B_r(x_0) \sep \forall \lambda \in \Lambda \sep |f_\lambda(x) | \le K'
  .\]
\end{lemma}

We'll first prove the lemma by appeal to the Baire category theorem and then
use it to prove the main statement of the theorem.

Let $\left( X, d \right) $ be a complete metric space and
$\left( f_\lambda \right)_{\lambda \in \Lambda} $ be a uniformly bounded family of continuous
functions. Define a family of closed sets $\left( A_k \right)_{k \in \N} $, where
\[
  A_k = \lbrace x \in X \st \forall \lambda \in \Lambda \sep | f_\lambda (x) | \le k \rbrace
.\]

Now in order to apply Baire category theorem we need to show that  all $A_k$ are
closed and that
\begin{equation}
X = \bigcup_{k = 1}^\infty A_k
.\end{equation}

First let us consider closedness. Note that for all $\lambda \in \Lambda$ the map
 $x \mapsto |f_\lambda(x)| $ is continuous by composition. That is because the
 absolute value function is continuous and we have also assumed that all  $f_\lambda$
 are continuous. Therefore if we rewrite eack of  $A_k$ as follows, we'll get:
  \[
    A_k = \lbrace x \in X \st \forall \lambda \in \Lambda \sep | f_\lambda (x) | \le k \rbrace = \bigcap_{\lambda \in \Lambda} \lbrace x \in X \st |f_\lambda(x)| \le k\rbrace
 .\]

 Now if we denote the map $g_\lambda := x \mapsto |f_\lambda(x)| $, then each of the
 sets of the form $\lbrace x \in X \st |f_\lambda(x)| \le k\rbrace$ is the
 preimage of  $g_\lambda$ of  $[0, k]$. Consequently it is the preimage of a
 continuous function of a closed set, hence it is closed. Therefore, each of the
 $A_k$ is an intersection of closed sets, and thus it is also closed.
 Hence we have established that all  $A_k$ are closed. Now in order to show that
 their union covers all of  $X$, note that we only need to show the forward insclusion
 i.e.
 \begin{equation}
    X \subseteq \bigcup_{k = 1}^\infty A_k
  .\end{equation}
 and the equation $(1)$ follows immediately from antisymmetry of set inclusion, because
 each of $A_k$ is already a subset of  $X$ and therefore the backward inclusion follows
 from the fact that a union of subsets is still a subset.

 In order to show the forward direction, let $x \in X$ be arbitrary. Now by our assumption
 of pointwise boundedness of the collection  $\left( f_\lambda \right)_{\lambda \in \Lambda} $,
for that particular $x$ we can find a  $K_x$ such that
$$\forall \lambda \in \Lambda \sep |f_\lambda\left( x \right)| \le K_x. $$

But now we can pick a sufficiently large $k \in \N$ such that  $k \ge K_x$ and then
we deduce:
\[
\forall \lambda \in \Lambda \sep |f_\lambda(x)| \le K_x \le k
.\]

And so now we can deduce that $x \in \bigcap_{\lambda \in \Lambda}\lbrace x \in X \st
|f_\lambda(x)| \le k \rbrace = A_k$. Which in turn implies that $x \in \bigcup_{k = 1}^\infty A_k$.
And since  $x$ was arbitrary, we can deduce that (2) holds which combined with
antisymmertry and the fact that the union is a subset of $X$ yields equation (1).

Now since $X$ is nonempty and complete, and given that  $X = \bigcup_{k = 1}^\infty A_k$, where
each  $A_k$ is closed, by Baire category theorem we deduce that there exists a
$k_0 \in \N$ such that the interior of  $A_{k_0}$ is non-empty.
Hence, because of the fact that $\int(A_{k_0})$ is open and non-empty, we can pick
$x_0$ inside of it and  $r > 0$ such that
 \[
   B_r(x_0) \subseteq A_{k_0}
.\]
Thus, by definition of $A_{k_0}$ we deduce that:
 \[
 \forall x \in B_r(x_0) \sep \forall \lambda \in \Lambda \sep |f_\lambda(x)| \le k_0
.\]
And so we have found our $K' := k_0$ whose existence we needed to show to prove
the lemma. \hfill \blacksquare

Having proved the lemma, we can now move on to proving the main theorem.
Let $X, Y$ be normed spaces, assume  $X$ is complete. Let
  $\left( A_\lambda \right)_{\lambda \in \Lambda} \subset \mathcal{L}\left( X, Y \right) $
  be a bounded pointwise family of linear operators. In order to prove that
  $\left( A_\lambda \right)_{\lambda \in \Lambda} $ is in fact uniformly bounded,
  let us first define a family of continuous functions to which we can then apply the
  previous lemma. For $\lambda \in \Lambda$ define a map  $f_\lambda : X \to \R$ as
   \[
   \forall x \in X \sep f_\lambda(x) = \|A_\lambda(x)\|_Y
  .\]
  Now note that the norm $y \mapsto \|y\|_Y$ is continuous and by our assumption
  $A_\lambda$ is a continuous linear map. Therefore, by composition we deduce that
  for all $\lambda \in \Lambda$  $f_\lambda $ is continuous and therefore we can
  apply the lemma to get a ball  $B = B_r(x_0)$ with  $r > 0$ and  $x_0 \in X$,
  such that  there exists $K' \in \R$ satisfying:
  \begin{equation}
 \forall x \in B_r(x_0) \sep \forall \lambda \in \Lambda \sep |f_\lambda(x)| \le K'
.\end{equation}
 Our objective is to show that
 \[
    \exists K'' \in \R \sucht \forall \lambda \in \Lambda \sep \|A_\lambda \|_{\mathcal{L}\left( X, Y \right) } \le K''
 .\]

 Now note that we are able to control the operator norm in a following way:
 \[
   \forall \lambda \in \Lambda \sep \forall x \in X \sep \|A_\lambda x\|_Y \le K'' \|x\|_X \implies \|A_\lambda\|_{\mathcal{L}\left( X, Y\right)} \le K''
 .\]

 Now fix $K'' := \dfrac{4K'}{r}$ where $K'$ is the constant bounding uniformly $|f_\lambda(s)|$
 that we have obtained from the lemma. We want to show that
 \[
   \forall \lambda \in \Lambda, \sep \|A_\lambda\|_{\mathcal{L}\left( X,Y \right)} \le K''
 .\]

 In order to do this, fix $\lambda \in \Lambda$ arbitrary and consider the following:
\[
   \forall x \in X \sep \|A_\lambda x\|_Y \le K'' \|x\|_X \implies \|A_\lambda\|_{\mathcal{L}\left( X, Y\right)} \le K''
.\]

Hence if we need to fix $x \in X$ arbitrary and impose a bound
\[
\|A_\lambda x\|_Y \le K'' \|x\|_X
.\]

In order to achieve this, consider the following chain of manipulations.
First note that by adding and subtracting $x_0$ and multiplying by a scaling factor
and its inverse, we get:
 \[
x = \frac{2\|x\|}{r}\left( x_0 + \frac{r}{2}\frac{x}{\|x\|} - x_0 \right)
.\]
It is important to note that in order for this to work we need $X$ to be a normed
vector space over $\R$ (as  $r \in \R$). And that is why in my Lean implementation
I took $X, Y$ to be vector spaces over  $\R$ as opposed to arbitrary fields  $\mathbb{F}_1, \mathbb{F}_2,$
Now if we substitute the expression above into $\|A_\lambda x\|$ we obtain:
\[
\|A_\lambda x\| = \left|\left|A_\lambda\left(\frac{2\|x\|}{r}\left( x_0 + \frac{r}{2}\frac{x}{\|x\|} - x_0 \right)  \right) \right|\right|
.\]
By linearity of $A_\lambda$ we can take the multiplicative factor out of the argument
and split the expression inside of the norm because $A_\lambda\left( \alpha x \right) = \alpha A_\lambda\left( x \right)  $ and
$A_\lambda\left( x - y \right) = A_\lambda\left( x \right)  - A_\lambda\left( y \right)  $
Hence, we obtain
\[
\left|\left|A_\lambda\left(\frac{2\|x\|}{r}\left( x_0 + \frac{r}{2}\frac{x}{\|x\|} - x_0 \right)  \right) \right|\right|_Y =
\left|\left| \frac{2\|x\|}{r} \left( A_\lambda\left( x_0 + \frac{r}{2}\frac{x}{\|x\|} \right) - A_\lambda(x_0)  \right) \right|\right|_Y
.\]
Now, by absolute homogeneity of $\|\cdot\|_Y$, we get:
\[
\left|\left| \frac{2\|x\|}{r} \left( A_\lambda\left( x_0 + \frac{r}{2}\frac{x}{\|x\|} \right) - A_\lambda(x_0)  \right) \right|\right|_Y =
\left| \frac{2\|x\|}{r} \right| \left|\left| \left( A_\lambda\left( x_0 + \frac{r}{2}\frac{x}{\|x\|} \right) - A_\lambda(x_0)  \right) \right|\right|_Y =
\]

Observe that both $0 \le r, \|x\|$, therefore we may drop the absolute value above
and apply the triangle inequality:
\begin{equation}
 \le \frac{2\|x\|}{r} \left( \left|\left| A_\lambda\left( x_0 + \frac{r}{2}\frac{x}{\|x\|} \right)
 \right|\right|_Y + \left|\left| A_\lambda(x_0)  \right|\right|_Y \right)
.\end{equation}

Now note that both $x_0 + \frac{r}{2}\frac{x}{\|x\|} $ and $x_0$ belong to $B_r(x_0)$.
In case of $x_0$ it is obvious, whereas when it comes to the first point, let us
consider the following to see that it is indeed contained in the ball  $B$:
\[
  x_0 + \frac{r}{2}\frac{x}{\|x\|} \in B_r{x_0} \iff \left|\left|x_0 - \left(x_0 + \frac{r}{2}\frac{x}{\|x\|}   \right) \right|\right| < r
.\]

Simplifying the left-hand side of the iff above allows us to see:
\[
\left|\left|x_0 - \left(x_0 + \frac{r}{2}\frac{x}{\|x\|}   \right) \right|\right| =
\left|\left| \frac{r}{2}\frac{x}{\|x\|}\right|\right| = \frac{r}{2} \frac{\|x\|}{\|x\|} = \frac{r}{2}  < r
.\]

Hence, we deduce that both of the points above belong to the open ball $B$, and
thus we can conclude:
\[
  \left|\left| A_\lambda\left( x_0 + \frac{r}{2}\frac{x}{\|x\|} \right)\right|\right|_Y = \left| f_\lambda\left( x_0 + \frac{r}{2}\frac{x}{\|x\|} \right)\right| \le K' \text{ and }
\left|\left| A_\lambda(x_0)  \right|\right|_Y = \left| f_\lambda(x_0)  \right| \le K'
.\]
That is because we can use the definition of $f_\lambda$ in reverse and then apply
 $(3)$ to impose those bounds.

 We can substitute the above findings back into the inequality $(4)$ to continue
 estimating the bound:
 \[
 \le \frac{2\|x\|}{r} \left( \left|\left| A_\lambda\left( x_0 + \frac{r}{2}\frac{x}{\|x\|} \right)
 \right|\right|_Y + \left|\left| A_\lambda(x_0)  \right|\right|_Y \right)
 \le \frac{2\|x\|}{r} \left( K' + K'\right) = \frac{4K'}{r}\|x\| = K''\|x\|
 .\]

 Since $\lambda$ and  $x$ were arbitrary and our choice of $K''$ didn't depend  either
 of them, we can conclude that:
\[
   \forall \lambda \in \Lambda \sep \forall x \in X \sep \|A_\lambda x\|_Y \le K'' \|x\|_X
.\]
Which in turn implies
\[
\forall \lambda \in \Lambda \sep \|A_\lambda\|_{\mathcal{L}\left( X, Y \right)}  \le K''.\]

And so the family $\left( A_\lambda \right)_{\lambda \in \Lambda} $ is uniformly
bounded and that concludes the proof. \hfill \blacksquare


\section*{The Process of Formalising}
\subsection*{Methodology}
\subsection*{Challenges}
\subsection*{Language Extensions}
\section*{Conclusions}
\section*{Future Work}




\end{document}
